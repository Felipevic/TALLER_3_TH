\section{Introducción}

El control de la resistencia del hormigón en obra es un aspecto fundamental para la gestión eficiente y segura del proceso constructivo. En este contexto, el Taller 3 busca aplicar diferentes metodologías de estimación de resistencia, relacionando la dosificación, el historial térmico y el tiempo de curado del material, con el fin de comprender su influencia en las decisiones técnicas y económicas que se adoptan durante la ejecución de estructuras de hormigón armado.

El taller se estructura en tres partes complementarias. En primer lugar, se estudia el método de madurez, el cual permite estimar la resistencia del hormigón a partir de su historial de temperatura interna. Este método, ampliamente utilizado en terreno, se basa en el principio de que una mezcla específica alcanza una misma resistencia cuando presenta el mismo índice de madurez, lo que permite determinar el tiempo equivalente necesario para realizar operaciones críticas, como el postensado o el retiro de moldajes.

En la segunda parte se aborda el análisis de la presión en los moldajes, considerando los factores que influyen en la carga lateral ejercida por el hormigón fresco, tales como la velocidad y la altura de colocación, la densidad y la temperatura de la mezcla. Este análisis permite comprender la importancia de una correcta planificación del vaciado y del diseño del sistema de encofrado para garantizar la seguridad estructural y la continuidad del proceso constructivo.

Finalmente, la tercera parte aplica los métodos empíricos de Bolomey y Venuat para estimar la resistencia del hormigón en función de su razón agua/cemento y del tiempo. El método de Bolomey permite determinar la dosis de cemento necesaria para alcanzar una resistencia objetivo a una edad específica, mientras que el de Venuat relaciona la resistencia con el tiempo de curado mediante una función logarítmica. Ambos métodos se aplican en un caso práctico orientado a optimizar el tiempo de desmolde y los costos de operación en columnas de obra gruesa, comparando alternativas con cemento corriente y de alta resistencia inicial.
