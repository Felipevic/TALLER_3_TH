\section{Desarrollo}

\subsection{Parte 1: Madurez}

En primer lugar se determina la madurez del hormigón según el método de Plowman, para luego determinar el tiempo necesario para alcanzar la resistencia requerida con los metodos de Nurse-Saul como la de Freieslaben, Hansen y Pedersen.

\subsubsection*{Plowman}

Considerando una temperatura de terreno de 23 °C ($T_r$), asi como $T_D$ tiene un valor de 0 °C, se calcula la madurez como:

\begin{equation}
    M = t_{eq} (T_r - T_D) \quad \text{donde} \quad t_{eq} = 13.27
\end{equation}

A partir de los datos entregados, se obtiene la siguiente tabla:

\begin{table}[htbp]
\centering
\renewcommand{\arraystretch}{1.15}
\begin{tabular}{r r r r r r r r}
\hline
\multicolumn{1}{c}{Edad} & \multicolumn{1}{c}{Temperatura} & \multicolumn{1}{c}{Resistencia} & \multicolumn{1}{c}{dt} & \multicolumn{1}{c}{T prom} & \multicolumn{1}{c}{Madurez} & \multicolumn{1}{c}{Madurez} & \multicolumn{1}{c}{Plowman} \\
\multicolumn{1}{c}{[días]} & \multicolumn{1}{c}{[°C]} & \multicolumn{1}{c}{MPa} & \multicolumn{1}{c}{días} & \multicolumn{1}{c}{C} & \multicolumn{1}{c}{C*h} & \multicolumn{1}{c}{C*dia} & \multicolumn{1}{c}{C*hr} \\
\hline
0 & 29 & 0 & 0 & -- & 0 & 0 & 0 \\
0.25 & 39 & 0.39 & 0.25 & 34 & 8.5 & 7 & 168 \\
0.5 & 49 & 0.88 & 0.25 & 44 & 19.5 & 14 & 336 \\
0.75 & 54 & 1.42 & 0.25 & 51.5 & 32.375 & 21 & 504 \\
1 & 59 & 2.01 & 0.25 & 56.5 & 46.5 & 28 & 672 \\
1.5 & 64 & 3.29 & 0.5 & 61.5 & 77.25 & 42 & 1008 \\
2 & 59 & 4.47 & 0.5 & 61.5 & 108 & 56 & 1344 \\
2.5 & 48 & 5.43 & 0.5 & 53.5 & 134.75 & 70 & 1680 \\
3 & 44 & 6.31 & 0.5 & 46 & 157.75 & 84 & 2016 \\
3.5 & 39 & 7.09 & 0.5 & 41.5 & 178.5 & 98 & 2352 \\
4 & 34 & 7.77 & 0.5 & 36.5 & 196.75 & 112 & 2688 \\
4.5 & 29 & 8.35 & 0.5 & 31.5 & 212.5 & 126 & 3024 \\
5 & 26 & 8.87 & 0.5 & 27.5 & 226.25 & 140 & 3360 \\
5.5 & 26 & 9.39 & 0.5 & 26 & 239.25 & 154 & 3696 \\
6 & 23 & 9.85 & 0.5 & 24.5 & 251.5 & 168 & 4032 \\
7 & 23 & 10.77 & 1 & 23 & 274.5 & 196 & 4704 \\
8 & 23 & 11.69 & 1 & 23 & 297.5 & 224 & 5376 \\
9 & 23 & 12.61 & 1 & 23 & 320.5 & 252 & 6048 \\
10 & 23 & 13.53 & 1 & 23 & 343.5 & 280 & 6720 \\
11 & 23 & 14.45 & 1 & 23 & 366.5 & 308 & 7392 \\
14 & 23 & 17.21 & 3 & 23 & 435.5 & 392 & 9408 \\
21 & 23 & 23.65 & 7 & 23 & 596.5 & 588 & 14112 \\
28 & 23 & 30.09 & 7 & 23 & 757.5 & 784 & 18816 \\
\hline
\end{tabular}
\end{table}

De esta manera, interpolando la madurez obtenida, se obtiene un tiempo de 8.33 dias, es decir, ese es el tiempo nesesario para el hormigon en obra tenga una resitencia similar a la resitencia alcanzada a los 13.27 dias en laboratorio.

Ahora, es nesesario determinar a madurez del hormigon al $85\%$ de $f_{cm}$, donde considerando un hormigon G20 con 4MPa de desviacion estandar, se obtiene:

\begin{equation}
    0.85 \cdot f_{cm} = f'_c + t \cdot s = (20 + 1.282 \cdot 4)\cdot 0.85 = 21.35 MPa
\end{equation}

Luego se obtienen los factores K1 y K2 a partir de los valores de 3 y 28 dias:

\begin{equation}
    6.31 = K_1 + K_2 log(2016)
\end{equation}

\begin{equation}
    30.09 = K_1 + K_2 log(18816)
\end{equation}

De esta forma, $K_1 = -74.698$ y $K_2 = 24.514$. Luego, la madurez para la resitencia requerida es:

\begin{equation}
    R(M) = K_1 + K_2 log(M)
\end{equation}

\begin{equation}
    21.35 = -74.698 + 24.514 log(M)
\end{equation}

De esta forma, la madurez requerida es de $M = 345.264$ C dia.

\subsubsection*{Nurse-Saul}

El metodo establece que el tiempo equivalente se puede calcular como:

\begin{equation}
    \Delta t_{eq} = \frac{T_i - T_d}{T_r - T_d} \Delta t_i
\end{equation}

De esta forma, se obtiene la siguiente tabla:

\begin{table}[H]
\centering
\renewcommand{\arraystretch}{1.15}
\begin{tabular}{r r r r r}
\hline
\multicolumn{1}{c}{Edad [días]} & \multicolumn{1}{c}{Temperatura [°C]} & \multicolumn{1}{c}{t eqi obra [días]} & \multicolumn{1}{c}{t equi lab [días]} & \multicolumn{1}{c}{Madurez c dia} \\
\hline
0 & 29 & 0 & 0 & 0 \\
0.25 & 39 & 0.423913043 & 0.39 & 7 \\
0.5 & 49 & 0.956521739 & 0.88 & 14 \\
0.75 & 54 & 1.543478261 & 1.42 & 21 \\
1 & 59 & 2.184782609 & 2.01 & 28 \\
1.5 & 64 & 3.576086957 & 3.29 & 42 \\
2 & 59 & 4.858695652 & 4.47 & 56 \\
2.5 & 48 & 5.902173913 & 5.43 & 70 \\
3 & 44 & 6.858695652 & 6.31 & 84 \\
3.5 & 39 & 7.706521739 & 7.09 & 98 \\
4 & 34 & 8.445652174 & 7.77 & 112 \\
4.5 & 29 & 9.076086957 & 8.35 & 126 \\
5 & 26 & 9.641304348 & 8.87 & 140 \\
5.5 & 26 & 10.20652174 & 9.39 & 154 \\
6 & 23 & 10.70652174 & 9.85 & 168 \\
7 & 23 & 11.70652174 & 10.77 & 196 \\
8 & 23 & 12.70652174 & 11.69 & 224 \\
9 & 23 & 13.70652174 & 12.61 & 252 \\
10 & 23 & 14.70652174 & 13.53 & 280 \\
11 & 23 & 15.70652174 & 14.45 & 308 \\
14 & 23 & 18.70652174 & 17.21 & 392 \\
21 & 23 & 25.70652174 & 23.65 & 588 \\
28 & 23 & 32.70652174 & 30.09 & 784 \\
\hline
\end{tabular}
\end{table}

Luego, interpolando la madurez obtenida, se obtiene un tiempo de 17.037 dias en obra y 15.674 dias en laboratorio para alcanzar la resitencia requerida de 21.35 MPa.

\subsection{Parte 3: Bolomey y Venuat}

En esta parte se requirió determinar el tiempo mínimo de desmolde de una columna para optimizar el presupuesto de la obra. Se compararon los tiempos de desmimbre utilizando un hormigón tradicional con uno de alta resistencia. 
Para el desarrollo de esta sección, se utilizaron los métodos de Bolomey y Venuat, los cuales permiten estimar la resistencia del hormigón en función de su dosificación y del tiempo de curado.

A continuación, se presentan los datos utilizados, los cálculos realizados para ambos tipos de hormigón y los resultados obtenidos.

\begin{table}[H]
\centering
\caption{Datos experimentales – Grupo 5}
\renewcommand{\arraystretch}{1.15}
\small
\begin{tabular}{lccc}
\hline
%\rowcolor[HTML]{EFEFEF}
\textbf{Parámetro} & \textbf{Unidad} & \textbf{Cem. corriente} & \textbf{Cem. alta resistencia} \\ \hline
Agua utilizada & kg/m$^3$ & 166.32 & 153.68 \\
Cemento mezcla 1 (Z1 / V1) & kg/m$^3$ & 432.23 & 324.68 \\
Cemento mezcla 2 (Z2 / V2) & kg/m$^3$ & 312.42 & 381.65 \\
Resistencia 14 días mezcla 1 & MPa & 20.45 & 24.24 \\
Resistencia 14 días mezcla 2 & MPa & 15.47 & 33.20 \\
Resistencia 28 días mezcla 1 & MPa & 27.27 & 29.55 \\
Resistencia 28 días mezcla 2 & MPa & 22.42 & 43.11 \\
Agua para columnas & kg/m$^3$ & 141 & 141\\ \hline
\end{tabular}
\end{table}

\subsubsection{Resistencia a la compresión}

En primer lugar, se determinó la resistencia requerida a partir de la especificada utilizando la siguiente fórmula: $f_{cm} = f_c + t \cdot s$, donde $f_c$ es la resistencia especificada, $t$ es un factor de seguridad y $s$ es la desviación estándar del hormigón.

\begin{table}[H]
\centering
\caption{Resistencia requerida para el desmolde}
\begin{tabular}{lcc}
\hline
Parámetro & Símbolo & Valor [MPa] \\ \hline
Desviación estándar & $s$ & 3.42 \\
Factor de seguridad & $t$ & 2.113 \\
Resistencia especificada & $f'_c$ & 28.86 \\
Resistencia requerida & $f_{cm}$ & 36.08 \\
Porcentaje requerido & \% & 90 \\
Resistencia mínima para desmolde & $f_{req}$ & 32.47 \\ \hline
\end{tabular}
\end{table}

\subsubsection{Dosis mínima de cemento}

En segundo lugar, se utilizó el método de Bolomey para determinar la dosis de cemento a utilizar para alcanzar la resistencia requerida. Se utilizó esta fórmula $R = a(\frac{c}{w} - b)$ para determinar los parámetros $a$, $b$, con los cuales se obtuvo la relación $c/w$ y finalmente de la dosis de cemento. Los sistemas de ecuaciones resueltos son los siguientes:

Para el cemento corriente:
\[
\begin{cases}
20.45 = a_{14}\,(3.07 - b_{14}) \\
15.47 = a_{14}\,(2.22 - b_{14})
\end{cases}
\]
\[
\begin{cases}
27.27 = a_{28}\,(3.07 - b_{28}) \\
22.42 = a_{28}\,(2.22 - b_{28})
\end{cases}
\]

Para el cemento de alta resistencia:
\[
\begin{cases}
24.24 = a_{14}\,(2.31 - b_{14}) \\
33.20 = a_{14}\,(2.71 - b_{14})
\end{cases}
\]
\[
\begin{cases}
29.55 = a_{28}\,(2.31 - b_{28}) \\
43.11 = a_{28}\,(2.71 - b_{28})
\end{cases}
\]


\begin{table}[H]
\centering
\caption{Parámetros obtenidas por el método de Bolomey para cada tipo de cemento}
\renewcommand{\arraystretch}{1.15}
\small
\begin{tabular}{lcc}
\hline
Parámetro & Cemento corriente & Cemento alta resistencia \\ \hline
c/w - Mezcla 1 (Z1 / V1) & 3.07 & 2.31 \\
c/w - Mezcla 2 (Z2 / V2) & 2.22 & 2.71 \\
$a_{14}$ & 5.85 & 22.15 \\
$b_{14}$ & -0.42 & 1.21 \\
$a_{28}$ & 5.70 & 33.51 \\
$b_{28}$ & -1.72 & 1.42 \\
$c/w$ & 4.62 & 2.50 \\ 
$c$ & 650.12 [kg/m$^3$] & 352.10 [kg/m$^3$] \\ \hline
\end{tabular}
\end{table}

